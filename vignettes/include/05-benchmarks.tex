\section{Benchmarks}\label{sec:benchmarks}

The \pkg{ngram} package is designed to be very fast.  This performance comes
about from the usual performance/generality tradeoff.  
The \pkg{tau}~\citep{tau} package offers a general framework for constructing
n-grams from a text, via its \code{textcnt()} function.  This is much more
general than the tools in the \pkg{ngram} package, but it is also
significantly slower.  And many of those additional features can be achieved
by preprocessing the text.

In \pkg{ngram}, the use of \code{get.phrasetable(ngram(x, n=3))} roughly corresponds
to \code{textcnt(x, n=3, split=" ", method="string")} in \pkg{tau}.  Although
\code{get.phrasetable()} returns proportions in addition to counts, and in the
form of a more costly dataframe compared to \code{tau}'s vector of counts, we
are still able to achieve very good performance.
\begin{lstlisting}[language=rr]
library(rbenchmark)
library(tau)
library(ngram)

x <- ngram::rcorpus(50000)

reps <- 15
cols <- c("test", "replications", "elapsed", "relative")

benchmark(tau=textcnt(x, n=3, split=" ", method="string"), ngram=get.phrasetable(ngram(x, n=3)), replications=reps, columns=cols)
\end{lstlisting}
Evaluating this script gives:
\begin{Code}
   test replications elapsed relative
2 ngram           15   4.184    1.000
1   tau           15 122.877   29.368
\end{Code}

In fact, a good portion of the time in the \pkg{ngram} runs here is in converting the
internal \proglang{C} data structure over to an \proglang{R} one.  The original
purpose of the \pkg{ngram} package was merely amusement, babbling n-grams.
If we just compare the run times for this, the difference is striking:
\begin{lstlisting}[language=rr]
library(tau)
library(ngram)

x <- ngram::rcorpus(100000)

tautime <- system.time(pt1 <- textcnt(x, n=3, split=" ", method="string"))[3]
ngtime <- system.time(pt2 <- ngram(x, n=3))[3]

cat("tau: ", tautime, "\n")
cat("ngram: ", ngtime, "\n")
cat("tau/ngram: ", tautime/ngtime, "\n")
\end{lstlisting}
If we evaluate this, we see:
\begin{Code}
tau:  32.746 
ngram:  0.084 
tau/ngram:  389.8333 
\end{Code}
Here, \pkg{ngram} is primed for babbling, in that it has already stored all
``next words'', while \pkg{tau} only contains what we call the phrasetable of
3-grams.
