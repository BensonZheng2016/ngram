\section{Introduction}
\label{sec:intro}

An n-gram is an ordered sequence of n ``words'' taken from a body of
text.  For example, consider the string \code{A B A C A B B}. This is the 
``blood code'' for the video game Mortal Kombat for the Sega Genesis, but you 
can pretend it's a biological sequence or something boring if you prefer.  If we
examine the 2-grams (or bigrams) of this sequence, they are:
\\
\begin{Code}
A B, B A, A C, C A, A B, B B
\end{Code}
or without repetition:
\\
\begin{Code}
A B, B A, A C, C A, B B
\end{Code}

That is, we take the input string and group the ``words'' 2 at a time (because 
\code{n=2}).  If we form all of the n-grams and record the next ``words'' for 
each n-gram (and their frequency), then we can generate new text which has the 
same statistical properties as the input.

The \thispackage package an \R package for constructing n-grams and generating 
new text as described above.  It also contains a few utilities to aid in this 
process.  Additionally, the \C code underlying this library can be compiled as 
a standalone shared library.